%*************************************************************
\chapter{Design VS Computer}\label{ch:DesignVSComputer}
%*************************************************************

Wir arbeiten mit Dokumenten in zwei verschiedenen Welten: der elektronischen Welt des Computers und der physikalischen Welt am Schreibtisch. Jede Welt hat Vorteile und Nachteile, welche uns dazu bewegen die >>richtige<< für bestimmte Aufgaben zu wählen (siehe \autoref{tab:wellnerDokumente}).

\begin{table}
    \myfloatalign
\begin{tabularx}{\textwidth}{p{5cm}X}
    \toprule
	    \tableheadline{Elektronische Dokumente} & \tableheadline{Papierdokumente}
	     \\ \midrule
		\begin{itemize} 
			\item{schnell zum editieren}
			\item{schnelles kopieren}
			\item{schnelles senden}
			\item{schnelles freigeben}
			\item{schnelles ablegen}
			\item{schnelles abrufen}
			\item{erlaubt Stichwortsuche}
			\item{erlaubt \newline Rechtschreibprüfung}
			\item{erlaubt sofortige \newline Berechnungen}
		\end{itemize} &
		\begin{itemize} 
			\item{3-dimensional}
			\item{überall akzeptiert}
			\item{billig}
			\item{portabel}
			\item{geläufig}
			\item{hochauflösend}
			\item{einfach zu lesen}
			\item{fühlbar}
			\item{man kann beide \newline Hände \& Finger zum bearbeiten verwenden}
			\item{man kann mit einem Stift darauf kritzeln}	
		\end{itemize}
	\\  \bottomrule
\end{tabularx}
  \caption[Elektronische und Papier]{Gegenüberstellung der Eigenschaften von elektronischen Dokumenten und Papierdokumenten.}
  \label{tab:wellnerDokumente}
\end{table}

\medskip In mancher Hinsicht scheint es als wäre Papier bald obsolet. Manche prophezeien das papierlose Büro schon in wenigen Jahren. Das Problem dabei ist nur, dass die Leute Papier mögen. Studien belegen, dass der Papierkonsum in Büros seit 1970 um das 6-fache anstieg und derzeit jährlich um 20\% steigt. \citep{seybold:1992} Papier hat ebenso wie elektronische Dokumente Eigenschaften, die Menschen nicht aufgeben wollen. Das macht sie in Hinsicht auf computerbasierende Alternativen >>unverwüstlich<<. \citep{Luff:1992}

Wie wichtig Papier im Zusammenhang mit Skizzen für Designer ist, wurde bereits im Kapitel \nameref{ch:designTheorie} beschrieben.


\section{Skizzieren auf Papier}

\section{Skizzieren am Computer}

\subsection{Electronic Ink}

\subsection{Das Modus Problem}\label{sec:ModusProblem}

\section{Mixed Reality}

\section{Die Bedeutung von Gestik}

\section{Case Study - Digital and Paper Media}

\section*{Zusammenfassung}