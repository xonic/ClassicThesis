%*************************************************************
\chapter{CSCW \& Design}\label{ch:CSCWDesign}
%*************************************************************

	Der Begriff >>Computer Supported Cooperative Work<< (CSCW) bezeichnet ein multidisziplinäres Forschungsgebiet und existiert seit den frühen achtziger Jahren. Um herauszufinden, wie die Technik Menschen bei ihrer Zusammenarbeit unterstützen kann, organisierten Irene Greif und Paul Cashman im Jahre 1984 einen Workshop für Personen, die sich mit der Arbeitsweise von Menschen auseinandersetzten \citep{Grudin:1994}. Unter diesen Personen befanden sich Spezialisten aus verschiedenen wissenschaftlichen Bereichen, wie zum Beispiel Ökonomie, Sozialpsychologie, Anthropologie, Ethnologie und Pädagogik. Dieser Workshop war der Versuch der Techniker, Teamarbeit besser zu verstehen, um folglich unterstützende Technologien entwickeln zu können \citep{Grudin:1994, Rama:2006p245}. Seither hat CSCW sich zu einem nahezu riesigen wissenschaftlichen Forschungsgebiet entwickelt, dem sich heute unzählige Experten widmen. Trotzdem sind sich die Wissenschaftler nicht immer ganz einig bei der Definition des Begriffes CSCW. Die Bedeutung von >>Cooperative Work<< erscheint nicht eindeutig und führt häufig zu unterschiedlichen Interpretation seitens wissenschaftlicher Autoren \citep{Gerlicher:2007p241}. Einige setzen >>Cooperation<< gleich mit >>Collaboration<<, andere hingegen unterscheiden die beden Begriffe sehr strikt. Dillenbourg et al. definieren die Begriffe beispielsweise so: 
	
	\bigskip\emph{>>Cooperation and collaboration do not differ in terms of whether or not the task is distributed, but by virtue of the way in which it is divided; in cooperation the task is split (hierarchically) into independent subtasks; in collaboration cognitive processes may be (heterarchically) divided into intertwined layers. In cooperation, coordination is only required when assembling partial results, while collaboration is “...„ a coordinated, synchronous activity that is the result of a continued attempt to construct and maintain a shared conception of a problem.<<} \citep{Dillenbourg:1995}
	
	\bigskip Die Erkenntnisse, welche die Erforschung von CSCW liefert, werden dazu verwendet, sinnvolle >>Groupware<< zu konzipieren. Groupware bezeichnet also die technische Umsetzung, die auf CSCW basiert. Alle Systeme, Applikationen und Werkzeuge, die CSCW unterstützen, können somit unter dem Begriff Groupware zusammengefasst werden \citep{Koch2008, Gerlicher:2007p241}. Häufig werden diese Systeme auch als >>kollaborative Software<< bezeichnet \citep{Bannon:1990p244}. 
	
	

\begin{itemize}
	\item {CSCW: four characters in search of a context}
	\item {CSCW an Enterprise 2.0 - towards an integrated perspective}
	\item {A survey and comparison of CSCW groupware applications}
	\item {CSCW - kollaborative Systeme und Anwendungen}
\end{itemize}

\section{Klassifikation von CSCW}
\subsection{Asynchrone Systeme}

Unter asynchroner Groupware versteht man Systeme, die keine Echtzeitanforderungen erfüllen müssen. Das bedeutet, dass die Nutzung dieser Systeme im Normalfall zeitversetzt erfolgt. Die wohl bekannteste asynchrone Groupware Anwendung ist die digitale Übermittlung von Textnachrichten in Form einer \emph{E-Mail}. Diese Nachrichten können an und von mehreren Personen versendet, empfangen und weitergeleitet werden. Dabei müssen Absender und Empfänger nicht gleichzeitig online sein, denn die \emph{E-Mail} kann zu jedem beliebigen Zeitpunkt vom Empfänger abgerufen werden.

Eine andere, jedoch der \emph{E-Mail} sehr ähnliche Form asynchroner Groupware sind \emph{Newsgroups und Mailinglisten} \citep{Gerlicher:2007p241}. Sie dienen dem Nachritenaustausch in einer größeren Gruppe an Benutzern. \emph{Newsgroups} zeigen Nachrichten nur dann an, wenn ein Benutzer diese direkt anfordert, \emph{Mailinglisten} hingegen werden automatisch an all jene Personen weitergeleitet, welche die entsprechende Liste abonniert haben.

Auch \emph{Workflow-Management Systeme} werden von Gerlicher und Ansgar \citep{Gerlicher:2007p241} zur asynchronen Groupware gezählt, da sie die Arbeit unterschiedlicher Personen in einer Organisation regeln. Sie dienen der Steuerung arbeitsteiliger Prozesse und man bezeichnet sie auch Geschäftsprozess-Management-Systeme.

Das World Wide Web (WWW) ist ein \emph{Hypertext-basiertes System} und gehört ebenfalls zur asynchronen Groupware \citep{Gerlicher:2007p241}, da es einer sehr großen Anzahl an Personen ermöglicht, auf digitalem Wege Informationen auszutauschen. \emph{Hypertext-basierte Systeme} nutzen das Internet oder ein bestimmtes Intranet, um ihre Dienste über einen Webbrowser zugänglich zu machen. Solch eine Webschnittstelle kann prinzipiell für fast jede Art der asynchronen Groupware implementiert werden. 

Schlussendlich nennen Gerlicher und Ansgar noch Gruppenkalender als asynchrones Groupware System. Der Gruppenkalender ist ein Tool, das sehr häufig in größeren Unternehmen eingesetzt wird. Er ermöglicht die Terminplanung und Koordination von vielen Personen. Terminkonflikte werden automatisch erkannt und das System kann eigenständig jene Zeiträume finden, zu denen jeder erforderliche Teilnehmer eines Meetings frei zur Verfügung steht. Dies setzt jedoch eine gute Datenpflege aller Benutzer voraus und wird teilweise als unangenehmer Eingriff in die Privatsphäre empfunden \citep{Gerlicher:2007p241}.

\subsection{Synchrone Systeme}

Dem gegenüber stehen synchrone Groupware Systeme, die sich dadurch charakterisieren, dass Benutzer zeitgleich auf Daten und Informationen zugreifen \citep{Gerlicher:2007p241}. Ein Beispiel dafür sind elektronische Tafeln, denn sie erlauben mehreren Nutzern, die sich an unterschiedlichen Orten befinden, auf eine für alle sichtbare Fläche zu zeichnen. Häufig werden solche Tafeln bei Besprechungen eingesetzt, bei denen die Teilnehmer nicht im selben Raum sind. Dadurch wird es möglich, Konzepte und Ideen besser zu kommunizieren und für die anderen greifbarer zu machen.

\emph{Application Sharing Systeme} zählen ebenfalls zur synchronen Groupware und gestatten das Teilen von Drittapplikationen mit anderen Personen \citep{Gerlicher:2007p241}. Man bezeichnet diese Anwendungen im Englischen auch als \emph{Remote Desktop}\footnote{Zu Deutsch: Entfernter Schreibtisch. Gemeint ist damit die virtuelle Schreibtischfläche des Betriebssystems.}. Dadurch wird es möglich, mit geografisch distanzierten Kollegen an jedem beliebigen Programm zu kooperieren. Beide Benutzer sehen die Applikation, die auf einem der Computer läuft und können mit ihr interagieren. 

Weitere synchrone Systeme sind Video- und Multimediale Konferenzsysteme \citep{Gerlicher:2007p241}. Erstere übertragen Video und Audio an mehrere verteilte Computer und werden eingesetzt, um virtuelle Meetings abzuhalten. Die Teilnehmer befinden sich meist an verschiedenen Orten und können so fast genauso kommunizieren, als würden sie sich in einem Raum befinden. Letztere bieten zusätzlich die Möglichkeit, multimediale Inhalte (beispielsweise Präsentationen) an die Teilnehmer der Konferenz zu übertragen. Systeme zu Entscheidungsfindung sind oft ein integraler Bestandteil solcher Software und bieten Werkzeuge für Ideenfindung und Brainstorming. Über auf diesem Wege generierte Konzepte kann dann abgestimmt und so die Spreu vom Weizen getrennt werden. 

\emph{Chat-Systeme}, ebenfalls der synchronen Groupware zuzuordnen, sind weitgehend bekannt und werden sehr häufig eingesetzt. Benutzer können durch diese Anwendungen in Echtzeit untereinander Textnachrichten austauschen und so eine Konversation führen, als säßen sie sich gegenüber. 

\subsection{Multisynchrone Systeme}

Versionsverwaltungssysteme werden von Gerlicher und Ansgar als multisynchrone Systeme definiert \citep{Gerlicher:2007p241}. Häufig werden diese in Softwareprojekten und auch bei der Erstellung von umfangreichen Dokumenten, beispielsweise Büchern eingesetzt. Das System übernimmt die Verwaltung der Dateien und sichert die Konsistenz der Daten. Es wird dadurch möglich, dass mehrere Personen gleichzeitig an der selben Datei arbeiten. Tatsächlich haben aber alle Benutzer eine eigene Kopie der Datei und das System synchronisiert die Dateien selbständig. Ein weiterer großer Vorteil von Versionsverwaltungssystemen, ist die Nachvollziehbarkeit von Änderungen \citep{Gerlicher:2007p241} an einzelnen Dateien. Das System legt für jede neue Version einer Datei ein Backup ab, das zu einem späteren Zeitpunkt bei Bedarf wiederhergestellt werden kann. Es ist zu jedem Zeitpunkt klar, wer welche Änderungen an einer bestimmten Datei durchgeführt hat. 

\bigskip Es wird deutlich, dass CSCW und Groupware sehr weitläufig sind und viele verschiedene Bereiche betreffen. Daher ist es bis heute immer noch eine große Herausforderung geblieben, bedienbare, effiziente und intuitive Systeme zu entwickeln, die die Arbeit der Benutzer tatsächlich optimieren und keinen Mehraufwand verursachen. Im folgenden Abschnitt widmen wir uns den Problemfeldern, die damit zusammenhängen.

\section{Wichtige Aspekte bei Design von Groupware}
\begin{itemize}
	\item {Helping CSCW Applications Succeed: The Role of Mediators in the Context of Use}
	\item {How Can Human and Design Sciences Cooperate in CSCW?}
	\item {CSCW as a basis for interactive design semantics}
	\item {Developing CSCW Tools for Idea Finding - Empirical Results and Implications for Design}
	\item {Design for Design: Support for Creative Practice in CSCW in Design}
	\item {Anforderungen an interaktive Kooperationslandschaften für kreatives Arbeiten und erste Realisierungen}
	\item {Making Sense of Collaboration: The Challenge of Thinking Together in Global Design Teams}
	\item {Single Display Groupware: A Model for Co-present Collaboration}
	\item {Synergy: A Prototype Collaborative Environment to Support the Conceptual Stages of the Design Process}
	\item {Architecture of BEACH: The Software Infrastructure for Roomware Environments}
	\item {A Multiple Device Approach for Supporting Whiteboard-based Interactions}
	\item {Avoiding Interference: How People use Spatial Separation and Partitioning in SDG Workspaces}
	\item {Being Here: Designing for Distributed Hands-On Collaboration in Blended Interaction Spaces}
	\item {Beyond the Chalkboard: Computer Support for Collaboration and Problem Solving in Meetings}
\end{itemize}

\section{Warum Groupware unseren Erwartungen oft nicht gerecht wird}
\begin{itemize}
	\item {The Productivity Paradox: Why hasn't Information Technology Fulfilled its Promise?}
	\item {Why CSCW Applications Fail: Problems in the Design and Evaluation of Organizational Interfaces}
\end{itemize}

\section{Der positive Nutzen von gut umgesetzter Groupware}

\section*{Zusammenfassung}