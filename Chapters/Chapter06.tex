%*************************************************************
\chapter{Single- VS Groupdesign}\label{ch:SingleVSGroupDesign}
%*************************************************************

lorem ipsum

\section{was auch immer}

\begin{table}
    \myfloatalign
\begin{tabularx}{\textwidth}{p{5cm}X}
    \toprule
	    \tableheadline{Design Criteria} & \tableheadline{Reasons}
	     \\ \midrule
	\small{
    1) 
	Provide ways of conveying and supporting gestural communication.
	Gestures should be clearly visible, and should maintain their relation with objects within the work surface and with voice communication.} & \small{
	\begin{compactitem}
		%\setlength{\parskip}{-6pt}
		%\setlength{\topsep}{-6pt}
		%\setlength{\partopsep}{-6pt}
		\item gestures are a prominent action %\par
		\item gestures are typically made in relation to objects on the work surface %\par
		\item gestures must be seen if they are to be useful %\par
		\item gestures are often accompanied by verbal explanation 
	\end{compactitem} }
	\\ [-12pt] \hline
	\small{
    2) 
	Minimize the overhead encountered when storing information.} & \small{
	\begin{compactitem}
		\item only one person usually records information %\par
		\item other participants should not be blocked from continuing private or group work while information is being stored 
	\end{compactitem} }
	\\ [-12pt] \hline
	\small{
    3) 
	Convey the process of creating artifacts to express ideas.} & \small{ 
	\begin{compactitem}
		\item the process of creation is in itself a gesture that communicates information %\par
		\item speech is closely synchronized with the creation process %\par
		\item artifacts in themselves are often meaningless 
	\end{compactitem} }
	\\ [-12pt] \hline
	\small{
	4) 
	Allow seamless intermixing of work surface actions and functions} & \small{ 
	\begin{compactitem}
		\item a single action often combines aspects of listing, drawing and gesturing %\par
		\item writing and drawing alternates rapidly %\par
		\item actions often address several functions 
	\end{compactitem} }
	\\ [-12pt] \hline
	\small{
	5) 
	Enable all participants to share a common view of the work surface while providing simultaneous access and a sense of close proximity to it} & \small{ 
	\begin{compactitem}
		\item people do not see the same things when orientation differs %\par
		\item simultaneous activity is prevalent %\par
		\item close proximity to the work surface encourages simultaneous activity 
	\end{compactitem} }
	\\ [-12pt] \hline
	\small{
	6) 
	Facilitate the participants natural abilities to coordinate their collaborations} & \small{ 
	\begin{compactitem}
		\item people are skilled at coordinating communication %\par
		\item we do not understand the coordinating process well enough to mechanize it 
	\end{compactitem} }
	\\ [-12pt] \bottomrule
\end{tabularx}
  \caption[Tangs Designkriterien]{Tangs Designkriterien zur Erstellung von Multi-User Zeichenprogrammen.}
  \label{tab:tangDesignKriterien}
\end{table}

\section*{Zusammenfassung}
lorem ipsum